\documentclass[journal,12pt,twocolumn]{IEEEtran}
\usepackage{gensymb}
\usepackage{amsmath}
\usepackage{relsize}
\usepackage{graphicx}
\usepackage{multicol}
\setlength{\columnsep}{1cm}
\begin{document}

\title{Properties of Triangle}
\author{G V V Sharma}
\maketitle
\begin{enumerate}
 \item In a $\Delta$\textsl{ABC}, $\angle$A = 90\degree and \textsl{AD} is an altitutde. Complete the realtion
 \[\frac{\textsl{$BD$}}{\textsl{$BA$}} = \frac{\textsl{$AB$}}{(....)}\]
 \item \textsl{ABC} is a triangle, \textsl{P} is a point on \textsl{AB}, and \textsl{Q} is point on \textsl{AC} such that $\angle$\textsl{AQP}=$\angle$\textsl{ABC}. Complete the relation
 \[\frac{\mbox{area of $\Delta$\textsl{APQ}}}{\mbox{area of $\Delta$\textsl{ABC}}}=\frac{(...)}{\textsl{$AC^2$}}\]
 \item \textsl{ABC} is a triangle with $\angle$\textsl{B} greater than $\angle$\textsl{C}. \textsl{D} and \textsl{E} are points on \textsl{BC} such that \textsl{AD} is perpendicular to \textsl{BC} and \textsl{AE} is the bisector of angle \textsl{A}. Complete the relation
 \begin{equation*}
 \angle\textsl{DAE}
 =\frac{1}{2}[\,( )-\angle\textsl{C}\:]\,
 \end{equation*}
 \item Find the set of all real numbers \textsl{a} such that \textsl{$a^2+2a, 2a+3$} and \textsl{$a^2+3a+8$} are the sides of a triangle ?
 \item In a triangle \textsl{ABC}, if $\cot$\textsl{A},
$\cot$\textsl{B}, $\cot$\textsl{C} are in A.P., then \textsl{$a^2, b^2, c^2$}, are in which progression?
 \item A polygon of nine sides, each of length 2, is inscribed in a circle. What is the radius of the circle ?
 \item If the angles of a triangle are 30\degree and 45\degree and the included side is (\,$\sqrt{3}+1$)\,cms, then find the area of traingle ?
 \item If in a triangle \textsl{$ABC$}, \begingroup
 \normalsize
 \begin{equation*}
 \frac{2\cos\textsl{A}}{\textsl{a}}+\frac{\cos\textsl{B}}{\textsl{b}}+\frac{2\cos\textsl{C}}{\textsl{c}}=\frac{\textsl{a}}{\textsl{bc}}+\frac{\textsl{b}}{\textsl{ca}}
 \end{equation*}
 \endgroup then find the value of angle \textsl{A} ?
 \item In a triangle \textsl{ABC}, \textsl{AD} is the altitude from \item[~] \textsl{A}. Given \textsl{$b>c$}, \textsl{$\angle$C}=23\degree and \textsl{AD}=$\dfrac{abc}{b^2-c^2}$ then find $\angle$\textsl{B} ?
 \item A circle is inscribed in an equilateral triangle of side a. Find the area of any square inscribed in this circle ?
 \item In a triangle \textsl{ABC}, \textsl{a:b:c} = 4:5:6. Find the ratio of the radius of the circumference to that of the incircle ? 
 \item If the bisector of the angle \textsl{P} of a triangle \textsl{PQR} meets \textsl{QR} in \textsl{S}, then
 \begin{multicols}{2}
 \begin{itemize}
 \item[(a)] \textsl{QS=SR} \item[~]
 \item[(c)] \textsl{QS:SR=PQ:PR}
 \item[(b)] \textsl{QS:SR=PR:PQ}\item[~]
 \item[(d)] None of these
 \end{itemize}
 \end{multicols}
 \item From the top of a light-house 60 metres high with its base at the sea-level, the angle of depression of a boat is 15\degree. Find the distance of the boat from the foot of the light-house ? 
 \begin{multicols}{2}
 \begin{itemize}
 \item[(a)] $\left(\dfrac{\sqrt{3}-1}{\sqrt{3}+1}\right)$60metres
 \item[(c)] $\left(\dfrac{\sqrt{3}+1}{\sqrt{3}-1}\right)^2$metres
 \item[(b)] $\left(\dfrac{\sqrt{3}-1}{\sqrt{3}+1}\right)$60metres
 \item[~]
 \item[(d)] none of these
 \end{itemize}
 \end{multicols}
 \item In a triangle \textsl{ABC}, angle \textsl{A} is greater than angle \textsl{B}. If the measures of angles \textsl{A} and \textsl{B} satisfy the equation $3sinx-4sin^3x-k=0, 0<k<1$, then the measure of angle C is
 \begin{itemize}
 \begin{multicols}{4}
 \item[(a)] $\dfrac{\pi}{3}$ \item[(b)] $\dfrac{\pi}{2}$ \item[(c)] $\dfrac{2\pi}{3}$ \item[(d)] $\dfrac{5\pi}{6}$
 \end{multicols}
 \end{itemize}
 \item If the lengths of the sides of triangle are 3, 5, 7 then the largest angle of the triangle is
 \begin{itemize}
 \begin{multicols}{4}
 \item[(a)] $\dfrac{\pi}{2}$ \item[(b)] $\dfrac{5\pi}{6}$ \item[(c)] $\dfrac{2\pi}{3}$ \item[(d)] $\dfrac{3\pi}{4}$
 \end{multicols}
 \end{itemize}
 \item In a triangle \textsl{ABC}, \angle B = $\dfrac{\pi}{3}$ and \angle C = $\dfrac{\pi}{4}$. \item[~]\item[~]
 Let \textsl{D} divide \textsl{BC} internally in the ratio 1:3 \item[~]\item[~]then $\dfrac{\sin \angle BAD}{\sin \angle CAD}$ is equal to\item[~]
 \begin{itemize}
 \begin{multicols}{4}
 \item[(a)]$\dfrac{1}{\sqrt{6}}$ \item[(b)]$\dfrac{1}{3}$ \item[(c)]$\dfrac{1}{\sqrt{3}}$ \item[(d)]$\sqrt{\dfrac{2}{3}}$
 \end{multicols}
 \end{itemize}
 \item In a triangle \textsl{ABC}, $2ac\sin\dfrac{1}{2}\left(A-B+C\right)$=
 \begin{itemize}
 \begin{multicols}{2}
 \item[(a)] $a^2+b^2-c^2$ \item[(c)] $b^2-c^2-a^2$ \item[(b)] $c^2+a^2-b^2$ \item[(d)] $c^2-a^2-b^2$
 \end{multicols}
 \end{itemize}
 \item In a triangle \textsl{ABC}, let $\angle$C = $\dfrac{\pi}{2}$. If \textsl{r} is the inradius and \textsl{R} is the circumradius of the triangle, then $2\left(r+R\right)$ is equal to
 \begin{itemize}
 \begin{multicols}{2}
 \item[(a)]a+b \item[(c)]c+a \item[(b)]b+c
 \item[(d)]a+b+c
 \end{multicols}
 \end{itemize}
 \item A pole stands vertically inside a triangular park $\Delta ABC$. If the angle of elevation of the top of the pole from each corner of the park is same, then in $\Delta ABC$ the foot of the pole is at
 \begin{itemize}
 \begin{multicols}{2}
 \item[(a)] centroid \item[(c)] incentre \item[(b)] circumcentre \item[(d)] orthocentre
 \end{multicols}
 \end{itemize}
 \item A man from the top of a 100 metres high tower sees a car moving towards the tower at an angle of depression of 30\degree. After some time, the angle of depression becomes 60\degree. The distance (in metres) travelled by the car during this time is
 \begin{itemize}
 \begin{multicols}{2}
 \item[(a)] $100\sqrt{3}$ \item[~]\item[(c)] $100\sqrt{3}/3$ \item[(b)] $200\sqrt{3}/3$\item[~] \item[(d)] $200\sqrt{3}$
 \end{multicols}
 \end{itemize}
 \item Which of the following pieces of data does NOT uniquely determine an acute-angled triangle \textsl{ABC} (\textsl{R} being the radius of the circumcircle) ?
 \begin{itemize}
 \begin{multicols}{2}
 \item[(a)] $a, \sin A, \sin B$ \item[(c)] $a, \sin B, R$ \item[(b)] $a, b, c$ \item[(d)] $a, \sin A, R$
 \end{multicols}
 \end{itemize}
 \item If the angles of a triangle are in the ratio 4 : 1 : 1, then the ratio of the longest side to the perimeter is
 \begin{itemize}
 \begin{multicols}{2}
 \item[(a)] $\sqrt{3} : \left(2+\sqrt{3}\right)$ \item[~]\item[(c)] $1 : 2+\sqrt{3}$ \item[(b)] $1 : 6$ \item[~]\item[(d)] $2 : 3$
 \end{multicols}
 \end{itemize}
 \item The sides of a triangle are in the ratio $1 : \sqrt{3} : 2$, then the angles of the triangle are in the ratio
 \begin{itemize}
 \begin{multicols}{4}
 \item[(a)] 1:3:5 \item[(b)] 2:3:4 \item[(c)] 3:2:1 \item[(d)] 1:2:3
 \end{multicols}
 \end{itemize}   
 \item In an equilateral triangle, 3 coins of radii 1 unit each are kept so that they touch each other and aslo the sides of the triangle. Area of the triangle is
 \begin{figure}[h!]
 \begin{center}
 \includegraphics[scale=0.77]{triangle.jpeg}
 \end{center}
 \end{figure}  
 \begin{itemize}
 \begin{multicols}{2}
 \item[(a)] $4+2\sqrt{3}$ \item[~]\item[(c)] 12+$\dfrac{7\sqrt{3}}{4}$ \item[(b)] $6+4\sqrt{3}$ \item[~]\item[(d)] 3+$\dfrac{7\sqrt{3}}{4}$
 \end{multicols}
 \end{itemize}
 \item In a triangle \textsl{ABC, a, b, c} are the lengths of its sides and \textsl{A, B, C} are the angles of triangle \textsl{ABC}. The correct relation is given by
 \begin{itemize}
 \item[(a)] $\left(b-c\right)\sin\left(\dfrac{B-C}{2}\right)=a\cos\dfrac{A}{2}$
 \item[~]
 \item[(b)] $\left(b-c\right)\cos\left(\dfrac{A}{2}\right)=a\sin\dfrac{B-C}{2}$
 \item[~]
 \item[(c)] $\left(b+c\right)\sin\left(\dfrac{B+C}{2}\right)=a\cos\dfrac{A}{2}$
 \item[~]
 \item[(d)] $\left(b-c\right)\cos\left(\dfrac{A}{2}\right)=2a\sin\dfrac{B+C}{2}$
 \end{itemize}
 \item One angle of an isosceles $\Delta$ is 120\degree and radius of its incircle=$\sqrt{3}$. Then the area of the triangle in sq. units is
 \begin{itemize}
 \begin{multicols}{2}
 \item[(a)] $7+12\sqrt{3}$ \item[~]\item[(c)] $12+7\sqrt{3}$ \item[(b)] $12-7\sqrt{3}$\item[~] \item[(d)] $4\pi$
 \end{multicols}
 \end{itemize}
 \item Let \textsl{ABCD} be a quadrilateral with area 18, with side \textsl{AB} parallel to the side \textsl{CD} and \textsl{2AB=CD}. Let \textsl{AD} be perpendicular to \textsl{AB} and \textsl{CD}. If a circle is drawn inside the quadrilateral \textsl{ABCD} touching all the sides, then its radius is
 \begin{itemize}
 \begin{multicols}{4}
 \item[(a)] 3 \item[(b)] 2 \item[(c)] $\dfrac{3}{2}$ \item[(d)] 1
 \end{multicols}
 \end{itemize}
 \item If the angles A, B and C of a triangle are in an arithmetic progression and if a, b and c denote the lengths of the side opposite to A, B and C respectively, then the value of \item[~] \item[~] the expression $\dfrac{a}{c}\sin2C+\dfrac{c}{a}\sin2A$ is
 \begin{itemize}
 \begin{multicols}{4}
 \item[(a)] $\dfrac{1}{2}$ \item[(b)] $\dfrac{\sqrt{3}}{2}$ \item[(c)] 1 \item[(d)] $\sqrt{3}$
 \end{multicols}
 \end{itemize}
 \item Let \textsl{PQR} be a triangle of area $\Delta$ with a=2, b=$\dfrac{7}{2}$, c=$\dfrac{5}{2}$, where \textsl{a, b, c} are the lengths \item[~] \item[~]of the sides of the triangle opposite to the angles at \textsl{P, Q} and \textsl{R} respectively. Then \item[~]$\dfrac{2\sin P-\sin 2P}{2\sin P+\sin 2P}$ equals
 \begin{itemize}
 \begin{multicols}{2}
 \item[(a)] $\dfrac{3}{4\Delta}$ \item[~]\item[(c)] $\left(\dfrac{3}{4\Delta}\right)^2$\item[(b)] $\dfrac{45}{4\Delta}$  \item[~]\item[(d)] $\left(\dfrac{45}{4\Delta}\right)^2$
 \end{multicols}
 \end{itemize}
 \item In a triangle the sum of two sides is x and the product of the same sides is y. If $x^2-c^2=y$, where c is the third side of the triangle, then the ratio of the in radius to the circum-radius of the triangle is
 \begin{itemize}
 \begin{multicols}{2}
 \item[(a)] $\dfrac{3y}{2x\left(x+c\right)}$ \item[~] \item[(c)] $\dfrac{3y}{4x\left(x+c\right)}$ \item[(b)] $\dfrac{3y}{2c\left(x+c\right)}$ \item[~] \item[(d)] $\dfrac{3y}{4c\left(x+c\right)}$
 \end{multicols}
 \end{itemize}
 \item There exists a triangle \textsl{ABC} satisfying the conditions
 \begin{itemize}
 \item[(a)] $b\sin A=a, A<\pi/2$
 \item[(b)] $b\sin A>a, A>\pi/2$
 \item[(c)] $b\sin A>a, A<\pi/2$
 \item[(d)] $b\sin A<a, A<\pi/2, b>a$
 \item[(e)] $b\sin A<a, A>\pi/2, b=a$
 \end{itemize}
 \item In a trinagle, the lengths of the two larger sides are 10 and 9 respectively. If the angles are in A.P. Then the length of the third side can be
 \begin{itemize}
 \begin{multicols}{2}
 \item[(a)] $5-\sqrt{6}$ \item[(c)] 5 \item[(e)] none \item[(b)] $3\sqrt{3}$ \item[(d)] $5+\sqrt{6}$
 \end{multicols}
 \end{itemize}
 \item If in a triangle \textsl{PQR}, $\sin P, \sin Q, \sin R$ are in A.P., then
 \begin{itemize}
 \begin{multicols}{2}
 \item[(a)] the altitudes are in A.P. \item[(c)] the medians are in G.P. \item[(b)] the altitudes are in H.P. \item[(d)] the medians are in A.P.
 \end{multicols}
 \end{itemize}
 \item Let \textsl{$A_0A_1A_2A_3A_4A_5$} be a regular hexagon inscribed in a circle of unit radius. Then the product of the lengths of the line segments \textsl{$A_0A_1, A_0A_2$} and \textsl{$A_0A_4$} is
 \begin{itemize}
 \begin{multicols}{2}
 \item[(a)] $\dfrac{3}{4}$ \item[~] \item[(c)] 3 \item[(b)] $3\sqrt{3}$ \item[~] \item[(d)] $\dfrac{3\sqrt{3}}{2}$
 \end{multicols}
 \end{itemize}
 \item In $\Delta ABC$, internal angle bisector of \angle A meets side \textsl{BC} in \textsl{D}. \textsl{DE $\perp$ AD} meets \textsl{AC} in \textsl{E} and \textsl{AB} in \textsl{F}. Then
 \begin{itemize}
 \item[(a)] AE is HM of b\& c \item[~]\item[(b)] $AD=\dfrac{2bc}{b+c} \cos \dfrac{A}{2}$ \item[~]\item[(c)] $EF=\dfrac{4bc}{b+c} \sin \dfrac{A}{2}$  \item[~] \item[(d)] $\Delta$AEF is isosceles
 \end{itemize}
 \item Let ABC be a triangle such that \angle ACB=$\dfrac{\pi}{6}$ and let a, b and c denote the lengths of the sides opposite to A, B and C respectively. The value(s) of x for which $a=x^2+x+1, b=x^2+1$ and $c=2x+1$ is(are)
 \begin{itemize}
 \begin{multicols}{2}
 \item[(a)] $-\left(2+\sqrt{3}\right)$\item[~] \item[(c)] $2+\sqrt{3}$ \item[(b)] $1+\sqrt{3}$\item[~] \item[(d)] $4\sqrt{3}$
 \end{multicols}
 \end{itemize}
 \item In a triangle \textsl{PQR, P} is the largest angle and $\cos P=\dfrac{1}{3}$. Further the incircle of the triangle touches the sides \textsl{PQ, QR} and \textsl{RP} at \textsl{N, L} and \textsl{M} respectively, such that the lengths of \textsl{PN, QL} and \textsl{RM} are consecutive even integers. Then possible length(s) of the side(s) of the triangle is(are)
 \begin{itemize}
 \begin{multicols}{2}
 \item[(a)] 16 \item[(c)] 24 \item[(b)] 18 \item[(d)] 22
 \end{multicols}
 \end{itemize}
 \item In a traingle XYZ, let x, y, z be the lengths of sides opposite to the angles X, Y, Z respectively, and $2s=x+y+z$. If $\dfrac{s-x}{4}=\dfrac{s-y}{3}=\dfrac{s-z}{2}$ and area of incircle of the traingle XYZ is $\dfrac{8\pi}{3}$, then
 \begin{itemize}
 \item[(a)] area of the triangle XYZ is $6\sqrt{6}$ \item[~]
 \item[(b)] the radius of circumcircle of the triangle XYZ is $\dfrac{35}{6}\sqrt{6}$\item[~]
 \item[(c)] $\sin\dfrac{X}{2}\sin\dfrac{Y}{2}\sin\dfrac{Z}{2}=\dfrac{4}{35}$ \item[~]
 \item[(d)] $\sin^2\left(\dfrac{X+Y}{2}\right)=\dfrac{3}{5}$
 \end{itemize}
 \item  In a triangle PQR, let \angle PQR=30\degree and the sides PQ and QR have lengths 10$\sqrt{3}$ and 10, respectively. Then, which of the following statement(s) is(are) TRUE ?
 \begin{itemize}
 \item[(a)] $\angle QPR=45\degree$
 \item[(b)] The are of the triangle PQR is $25\sqrt{3}$ and \angle QRP=120\degree
 \item[(c)] The radius of the incircle of the triangle PQR is $10\sqrt{3}-15$
 \item[(d)] The area of the circumcircle of the triangle PQR is 100$\pi$
 \end{itemize}
 \item In a non-right angled triangle $\Delta PQR$, let p, q, r denote the lengths of the sides opposite to the angles at P, Q, R respectively. The median from R meets the side PQ at S, the perpendicular from P meets the side QR at E, RS and PE intersect at O. If $p=\sqrt{3}, q=1$ and the radius of the circumcircle of the $\Delta PQR$ equals 1, then which of the following options is/are correct ?
 \begin{itemize}
 \item[(a)] Radius of incircle of $\Delta PQR=\dfrac{\sqrt{3}}{2}\left(2-\sqrt{3}\right)$
 \item[(b)] Area of $\Delta SOE=\dfrac{\sqrt{3}}{12}$
 \item[(c)] Length of OE=$\dfrac{1}{6}$
 \item[(d)] Length of RS=$\dfrac{\sqrt{7}}{2}$
 \end{itemize}
 \item A triangle $ABC$ has sides \textsl{AB=AC=}5 cm and \textsl{BC}=6 cm Triangle $A'B'C'$ is the reflection of the triangle $ABC$ in a line parallel to $AB$ placed at a distance 2 cm from $AB$, outside the triangle $ABC$. Triangle $A''B''C''$ is the reflection of the triangle $A'B'C'$ in a line parallel to $B'C'$ placed at a distance of 2 cm from $B'C'$ outside the triangle $A'B'C'$. Find the distance between $A$ and $A''$.
 \item \begin{itemize}
 \item[(a)] If a circle is inscribed in a right angled traingle $ABC$ with the right angle at $B$, show that the diameter of the circle is equal to $AB+BC-AC$.
 \item[(b)] If a triangle is inscribed in a cricle, then the product of any two sides of the triangle is equal to the product of the diameter and the perpendicular distance of the third side from the oppposite vertex. Prove the above statement.
 \end{itemize}
 \item \begin{itemize}
 \item[(a)] A balloon is observed simultaneously from three points \textsl{A, B} and \textsl{C} on a straight road directly beneath it. The angular elevation at \textsl{B} is twice that at \textsl{A} and the angular elevation at \textsl{C} is thrice at \textsl{A}. If the distance between \textsl{A} and \textsl{B} is a and the distance betweeen \textsl{B} and \textsl{C} is b, find the height of the balloon in terms of a and b.
 \item[(b)] Find the area of the smaller part of a disc of radius 10 cm, cut off by a chord \textsl{AB} which subtends an angle of $22\dfrac{1}{2}\degree$ at the circumference.
 \end{itemize}
 \item \textsl{ABC} is a triangle. \textsl{D} is the middle point of \textsl{BC}. If \textsl{AD} is perpendicular to \textsl{AC}, then prove that
 \begin{equation*}
 \cos A \cos C=\dfrac{2\left(c^2-a^2\right)}{3ac}
 \end{equation*}
 \item \textsl{ABC} is a triangle with \textsl{AB=AC}. \textsl{D} is any point on the side \textsl{BC}. \textsl{E} and \textsl{F} are points on the side \textsl{AB} and \textsl{AC}, respectively, such that \textsl{DE} is parallel to \textsl{AC}, and \textsl{DF} is parallel to \textsl{AB}. Prove that
 \begin{equation*}
 DF+FA+AE+ED=AB+AC
 \end{equation*}
 \item \begin{itemize}
 \item[(i)] \textsl{PQ} is a vertical tower. \textsl{P} is the foot and \textsl{Q} is the top of the tower, \textsl{A, B, C} are three points in the horizontal plane through \textsl{P}. The angles of elevation of \textsl{Q} from \textsl{A, B, C} are equal, and each is equal to $\theta$. The sides of the traingle \textsl{ABC} are \textsl{a, b, c}; and the area of the triangle \textsl{ABC} is $\Delta$. Show that the height of the tower is $\dfrac{abc\tan\theta}{4\Delta}$. 
 \item[(ii)] \textsl{AB} is a vertical pole. The end \textsl{A} is on the level ground. \textsl{C} is the middle point of \textsl{AB}. \textsl{P} is a point on the level ground. The portion \textsl{CB} subtends an angle $\beta$ at \textsl{P}. If \textsl{AP=nAB}, then show that $$\tan\beta=\dfrac{n}{2n^2+1}$$ 
 \end{itemize}
 \item Let the angles \textsl{A, B, C} of a triangle \textsl{ABC} be in A.P. and let $b:c=\sqrt{3}:\sqrt{2}$. Find the angle \textsl{A}.
 \item A vertical pole stands at a point \textsl{Q} on a horizontal ground, \textsl{d} meters apart. The pole subtends angles $\alpha$ and $\beta$ at \textsl{A} and \textsl{B} respectively. \textsl{AB} subtends an angle $\gamma$ at \textsl{Q}. Find the height of the pole.
 \item Four ships \textsl{A, B, C} and \textsl{D} are at sea in the following relative positions : \textsl{B} is on the straight line segment \textsl{AC, B} is due North of \textsl{D} and \textsl{D} is due west of \textsl{C}. The distance between \textsl{A} and \textsl{D} is 2 km. \angle BDA = 40\degree, \angle BCD = 25\degree. What is the distance between \textsl{A} and \textsl{D} ? [Take $\sin$25\degree = 0.423] 
 \item The ex-radii \textsl{$r_1, r_2, r_3$} of $\Delta$ \textsl{ABC} are in H.P. Show that its sides \textsl{a, b, c} are in A.P.
 \item For a traingle \textsl{ABC} it is given that $\cos A+\cos B+\cos C=\dfrac{3}{2}$. Prove that the traingle is equilateral.
 \item With usual notation, if in a traingle \textsl{ABC};
 $\dfrac{b+c}{11}=\dfrac{c+a}{12}=\dfrac{a+b}{13}$ then prove that \item[~] \item[~]$\dfrac{\cos A}{7}=\dfrac{\cos B}{19}=\dfrac{\cos C}{25}$.
 \item A ladder rests against a wall at an angle $\alpha$ to the horizontal. Its foot is pulled away from the wall through a distance \textsl{a}, so that it slides \textsl{a} distance \textsl{b} down the wall making an angle $\beta$ with the horizontal. Show that $a=b\tan\dfrac{1}{2}\left(\alpha+\beta\right)$
 \item In a traingle \textsl{ABC}, the median to the side \textsl{BC} is of length $\dfrac{1}{\sqrt{11-6\sqrt{3}}}$ and it divides \item[~] \item[~]the angle \textsl{A} into angles 30\degree adn 45\degree. Find the length of the side \textsl{BC}.
 \item If in a triangle \textsl{ABC}, $\cos A \cos B+\sin A \sin B \sin C = 1$, Show that $a:b:c = 1:1:\sqrt{2}$
 \item A sign-post in the form of an isosceles triangle \textsl{ABC} is mounted on a pole of height h fixed to the ground. The base \textsl{BC} of the triangle is parallel to the ground. A man standing on the ground at a distance d from the sign-post finds the top vertex \textsl{A} of the traingle subtends an angle $\beta$ and either of the other two vertices subtends the same angle $\alpha$ at his feet. Find the area of the traingle.
 \item \textsl{ABC} is a traingular park with \textsl{AB = AC = }100 m. A telivision tower stands at the midpoint of \textsl{BC}. The angles of elevation of the top of the tower at \textsl{A, B, C} are 45\degree, 60\degree, 60\degree, respectively. Find the height of the tower.
 \item A vertical tower \textsl{PQ} stands at a point \textsl{P}. Points \textsl{A} and \textsl{B} are located to the South and East of \textsl{P} respectively. \textsl{M} is the mid point of \textsl{AB}. \textsl{PAM} is an equilateral triangle; and N is the foot of the perpendicular from \textsl{P} on \textsl{AB}. Let \textsl{AN} = 20 metres and the angle of elevation of the top of the tower at \textsl{N} is $\tan^{-1}\left(2\right)$. Determine the height of the tower and the angles of elevation of the top of the tower at \textsl{A} and \textsl{B}.
 \item The sides of a triangle are three consecutive natural numbers and its largest angle is twice the smallest one. Determine the sides of the traingle.
 \item In a triangle of base $'a'$ the ratio of the other two sides is $r\left(<1\right)$. Show that the altitude \item[~] \item[~] of the triangle is less than or equal to $\dfrac{ar}{1-r^2}$. \item[~] 
 \item A man notices two objects in a straight line due west. After walking a distance $c$ due north he observes that the objects subtend an angle $\alpha$, at his eye; and, after walking a further distance 2$c$ due north, an angle $\beta$. Show that the distance between the objects \item[~]is $\dfrac{8c}{3\cot\beta-\cot\alpha}$; the height of the man is \item[~]\item[~]being ignored.
 \item Three circles touch the one another externally. The tangent at their point of contact meet at a point whose distance from a point of contact is 4. Find the ratio of the product of the radii to the sum of the radii of the circles.
 \item An observer at \textsl{O} notices that the angle of elevation os the top of a tower is 30\degree. The line joining \textsl{O} to the base of the tower makes an angle of $\tan^{-1}\left(1/\sqrt{2}\right)$ with the North and is inclined Eastwards. The observer travels a distance of 300 meters towards the North to a point A and finds the tower to his East. The angle of elevation of the top of the tower \textsl{A} is $\phi$, Find $\phi$ and the height of the tower.
 \item A tower \textsl{AB} leans towards west making an angle $\alpha$ with the vertical. The angular elevation of \textsl{B}, the topmost point of the tower is $\beta$ as observed from a point \textsl{C} due west of \textsl{A} at a distance \textsl{d} from \textsl{A}. If the angular elevation of \textsl{B} from a point \textsl{D} due east of \textsl{C} at a distance \textsl{2d} from \textsl{C} is $\gamma$, then prove that $2\tan\alpha=-\cot\beta+\cot\gamma$
 \item Let \textsl{$A_1, A_2,......, A_n$} be the vertices of an \textsl{n}-sided regular polygon such that \item[~]$\dfrac{1}{A_1A_2}=\dfrac{1}{A_1A_3}+\dfrac{1}{A_1A_4}$, Find the value of \item[~] \item[~] \textsl{n}.  
 \item Consider the following statements concerning a traingle \textsl{ABC}
 \begin{itemize}
 \item[(i)] The sides \textsl{a, b, c} and area $\Delta$ are rational. \item[~]
 \item[(ii)] $a, \tan\dfrac{B}{2}, \tan\dfrac{C}{2}$ are rational. \item[~]
 \item[(iii)] $a, \sin A, \sin B, \sin C$ are rational. 
 \end{itemize}\item[~]
 Prove that (i)$\implies$(ii)$\implies$(iii)$\implies$(i)
 \item A bird flies in a circle on a horizontal plane. An observer stands at a point on the ground. Suppose 60\degree and 30\degree are the maximum and minimum angles of elevation of the bird and that they occur when the bird is at the points \textsl{P} and \textsl{Q} respectively on its path. Let $\theta$ be the angle of elevation of the bird when it is a point on the arc of the circle exactly midway between \textsl{P} and \textsl{Q}. Find the numerical value of $\tan^2\theta$. (Assume that the observer is not inside the vertical projection of the path of the bird.)
 \item Prove that a triangle \textsl{ABC} is equilateral if and only if $\tan A+\tan B+\tan C = 3\sqrt{3}$.
 \item Let \textsl{ABC} be a triangle having \textsl{O} and \textsl{I} as its circumcentre and in centre respectively. If \textsl{R} and \textsl{r} are the circumradius and inradius, respectively, then prove that $\left(IO\right)^2 = R^2-2Rr$. Further show that the traingle BIO is a right-angled triangle if and only if \textsl{b} is arithmetic mean of \textsl{a} and \textsl{c}.
 \item Let \textsl{ABC} be a triangle with incentre \textsl{I} and inradius \textsl{r}. Let \textsl{D, E, F} be the feet of perpendiculars from \textsl{I} to the sides \textsl{BC, CA} and \textsl{AB} respectively. If $r_1, r_2$ and $r_3$ are the radii of circles inscribed in the quadrilaterals \textsl{AFIE, BDIF} and \textsl{CEID} respectively, prove that
 $\dfrac{r_1}{r-r_1}+\dfrac{r_2}{r-r_2}+\dfrac{r_3}{r-r_3}=$\item[~] \item[~]$\dfrac{r_1r_2r_3}{\left(r-r_1\right)\left(r-r_2\right)\left(r-r_3\right)}$ \item[~]
 \item If $\Delta$ is the area of a triangle with side lengths \textsl{a, b, c}, then show that $\Delta\leq\dfrac{1}{4}\sqrt{\left(a+b+c\right)abc}$. Also show that the equality occurs in the above inequality if and only if $a=b=c$.
 \item If \textsl{$I_n$} is the area of \textsl{n} sided regular polygon inscribed in a circle of unit radius and \textsl{$O_n$} be the area of the polygon circumscribing the given circle, prove that
 $$I_n = \dfrac{O_n}{2}\left(1+\sqrt{1-\left(\dfrac{2I_n}{n}\right)^2}\right)$$.
 \item Let \textsl{ABC} and \textsl{$ABC '$} be two non-congruent triangles with sides $AB = 4, AC = AC' = 2\sqrt{2}$ and angle $B = 30$\degree. Find the absolute value of the difference between the areas of these triangles ? 
 \item Consider a triangle \textsl{ABC} and let \textsl{a, b} and \textsl{c} denote the lengths of the sides opposite to vertices \textsl{A, B} and \textsl{C} respectively. Suppose $a = 6, b = 10$ and the area of the triangle is $15\sqrt{3}$, if \angle ACB is obtuse and if \textsl{r} denotes the radius of the incircle of the triangle, then $r^2$ is equal to ?
 \item The sides of a triangle are $3x+4y, 4x+3y$ and $5x+5y$ where $x, y>0$ then the triangle is
 \begin{itemize}
 \item[(a)] right angled
 \item[(b)] obtuse angled
 \item[(c)] equilateral
 \item[(d)] none of these
 \end{itemize}
 \item In a triangle with sides $a, b, c, r_1>r_2>r_3$ (which are the exradii) then
 \begin{itemize}
 \begin{multicols}{2}
 \item[(a)] $a>b>c$ \item[(c)] $a>b$ and $b<c$ \item[(b)] $a<b<c$ \item[(d)] $a<b$ and $b>c$
 \end{multicols}
 \end{itemize}
 \item The sum of the radii of inscribed and circumscribed circles for an n sided regular polygon of side a, is
 \begin{itemize}
 \begin{multicols}{2}
 \item[(a)] $\dfrac{a}{4}\cot\left(\dfrac{\pi}{2n}\right)$ \item[~] \item[~]
 \item[(c)] $\dfrac{a}{2}\cot\left(\dfrac{\pi}{2n}\right)$
 \item[(b)] $a\cot\left(\dfrac{\pi}{n}\right)$ \item[~] \item[~]
 \item[(d)] $a\cot\left(\dfrac{\pi}{2n}\right)$
 \end{multicols}
 \end{itemize}
 \item In a triangle ABC, medians AD and BE are drawn. If AD = 4, \angle DAB = $\dfrac{\pi}{6}$ and \angle ABE = $\dfrac{\pi}{3}$, then the area of the $\Delta ABC$ is
 \begin{itemize}
 \begin{multicols}{2}
 \item[(a)] $\dfrac{64}{3}$ \item[~] \item[(c)] $\dfrac{16}{3}$
 \item[(c)] $\dfrac{8}{3}$ \item[~] \item[(d)] $\dfrac{32}{3\sqrt{3}}$
 \end{multicols}
 \end{itemize}
 \item If in a $\Delta$ABC $a \cos^2\left(\dfrac{C}{2}\right)+c \cos^2\left(\dfrac{A}{2}\right)=\dfrac{3b}{2}$, then the sides \textsl{a, b} and \textsl{c} \item[~] \item[~]
 \begin{itemize}
 \item[(a)] satisfy $a+b = c$ \item[(b)] are in A.P. \item[(c)] are in G.P. \item[(d)] are in H.P.
 \end{itemize}
 \item The sides of a triangle are $\sin\alpha, \cos\alpha$ and\item[~]\item[~] $\sqrt{1+\sin\alpha \cos\alpha}$ for some $0<\alpha<\dfrac{\pi}{2}$. Then \item[~]\item[~]the greatest angle of the triangle is
 \begin{itemize}
 \begin{multicols}{4}
 \item[(a)] 150\degree \item[(b)] 90\degree \item[(c)] 120\degree \item[(d)] 60\degree
 \end{multicols}
 \end{itemize}
 \item A person standing on the bank of a river observes that the angle of elevation of the top of a tree on the opposite bank of the river is 60\degree and when he retires 40 meters away from the tree the angle of elevation becomes 30\degree. The breadth of the river is
 \begin{itemize}
 \begin{multicols}{4}
 \item[(a)] $60 m$ \item[(c)] $30 m$ \item[(b)] $40 m$ \item[(d)] $20m$
 \end{multicols}
 \end{itemize}
 \item In a triangle \textsl{ABC}, let \angle C = $\dfrac{\pi}{2}$. If \textsl{r} is the inradius and \textsl{R} is the circumradius of the \item[~]triangle \textsl{ABC}, then $2\left(r+R\right)$ equals
 \begin{itemize}
 \begin{multicols}{2}
 \item[(a)] b+c \item[(c)] a+b+c \item[(b)] a+b \item[(d)] c+a
 \end{multicols}
 \end{itemize}
 \item If in a $\Delta ABC$, the altitudes from the vertices \textsl{A, B, C} on opposite sides are in H.P, then $\sin A, \sin B, \sin C$ are
 \begin{itemize}
 \begin{multicols}{2}
 \item[(a)] G.P. \item[(b)] A.P - G.P \item[(b)] A.P \item[(d)] H.P
 \end{multicols}
 \end{itemize}
 \item A tower stands at the centre of a circular park. \textsl{A} and \textsl{B} are two points on the boundary of the park such that $AB (=a)$ subtends an angle of 60\degree at the foot of the tower, and the angle of elevation of the top of the tower from A or B is 30\degree. The height of the tower is
 \begin{itemize}
 \begin{multicols}{2}
 \item[(a)] $a/ \sqrt{3}$ \item[(c)] $2a/ \sqrt{3}$ \item[(b)] $a\sqrt{3}$ \item[(d)] $2a\sqrt{3}$
 \end{multicols}
 \end{itemize}
 \item \textsl{AB} is a vertical pole with \textsl{B} at the ground level and \textsl{A} at the top. A man finds that the angle of elevation of the point \textsl{A} from a certain point \textsl{C} on the ground is 60\degree. He  moves away from the pole along the line \textsl{BC} to a point \textsl{D} such that $CD = 7 m$. From \textsl{D} the angle of elevation of the point \textsl{A} is 45\degree. Then the height of the pole is
 \begin{itemize}
 \begin{multicols}{2}
 \item[(a)] $\dfrac{7\sqrt{3}}{2}\dfrac{1}{\sqrt{3}-1} m$ \item[~] \item[~]
 \item[(c)] $\dfrac{7\sqrt{3}}{2}\left(\sqrt{3}-1\right) m$
 \item[(b)] $\dfrac{7\sqrt{3}}{2}\left(\sqrt{3}+1\right) m$\item[~] \item[~]
 \item[(d)] $\dfrac{7\sqrt{3}}{2}\dfrac{1}{\sqrt{3}+1} m$
 \end{multicols}
 \end{itemize}  
 \item For a regular polygon, let r and R be the radii of the inscribed and the circumscribed circles. A \textsl{false} statement among the following is
 \begin{itemize}
 \item[(a)] There is a regular polygon with $\dfrac{r}{R}=\dfrac{1}{\sqrt{2}}$ \item[~]
 \item[(b)] There is a regular polygon with $\dfrac{r}{R}=\dfrac{2}{3}$ \item[~]
 \item[(c)] There is a regular polygon with $\dfrac{r}{R}=\dfrac{\sqrt{3}}{2}$ \item[~]
 \item[(d)] There is a regular polygon with $\dfrac{r}{R}=\dfrac{1}{2}$
 \end{itemize}  
 \item A bird is sitting on the top of a vertical pole 20 m high and its elevation from a point \textsl{O} on the ground is 45\degree. It flies off horizontally straight away from the point \textsl{O}. After one second, the elevation of the bird from \textsl{O} is reduced to 30\degree. Then the speed (in m/s) of the bird is
 \begin{itemize}
 \begin{multicols}{2}
 \item[(a)] $20\sqrt{2}$ \item[~]\item[(c)] $40\left(\sqrt{2}-1\right)$ \item[(b)] $20\left(\sqrt{3}-1\right)$ \item[~]\item[(d)] $40\left(\sqrt{3}-\sqrt{2}\right)$
 \end{multicols}
 \end{itemize}
 \item If the angles of elevation of the top of a tower from three collinear points A, B and C, on a line leading to the foot of the tower, are 30\degree, 45\degree and 60\degree respectively, then the ratio, AB : BC, is
 \begin{itemize}
 \begin{multicols}{2}
 \item[(a)] $1:\sqrt{3}$ \item[(c)] $\sqrt{3}:1$ \item[(b)] $2:3$ \item[(d)] $\sqrt{3}:\sqrt{2}$
 \end{multicols}
 \end{itemize}
 \item Let a vertical tower AB have its end A on the level ground. Let C be the mid-point of AB and P be a point on the ground such that AP = 2AB. If \angle BPC = $\beta$, then $\tan\beta$ is equal to
 \begin{itemize}
 \begin{multicols}{2}
 \item[(a)] $\dfrac{4}{9}$ \item[~]\item[(c)] $\dfrac{1}{4}$ \item[(b)] $\dfrac{6}{7}$ \item[~] \item[(d)] $\dfrac{2}{9}$
 \end{multicols}
 \end{itemize}
 \item PQR is a triangular park with PQ = PR = 200 m. A T.V tower stands at the mid-point of QR. If the angles of elevation of the top of the tower at P, Q and R are respectively 45\degree, 30\degree and 30\degree, then the height of the tower (in m) is
 \begin{itemize}
 \begin{multicols}{2}
 \item[(a)] 50 \item[(c)] $50\sqrt{2}$ \item[(b)] $100\sqrt{3}$ \item[(d)] 100
 \end{multicols}
 \end{itemize}
 \end{enumerate}
\end{document}